\documentclass[pts12]{article}

% Revisar TODOs.

\usepackage[utf8]{inputenc}
\usepackage[amsmath]{}
\usepackage[usenames,dvipsnames,svgnames,table]{xcolor}
\usepackage[tmargin=1in,bmargin=1in,lmargin=1.25in,rmargin=1.25in]{geometry}
\usepackage[rightcaption]{sidecap}
\usepackage{wrapfig}
\usepackage{graphicx}
\usepackage{color}
\usepackage{mathtools}
\usepackage{amssymb}
\usepackage{amsmath}
\usepackage{pifont}
\usepackage{eurosym}
\usepackage{hyperref}
% Use multiple columns when itemizing
\usepackage{multicol}
\usepackage[english]{babel}
\usepackage{graphicx}
\usepackage{twoopt}
\usepackage{hyperref}
\usepackage{verbatim}
% Spacing for equations in tables
\usepackage{tabu}

\graphicspath{ {images/} }

\numberwithin{equation}{section}

\title{Cálculo diferencial e interal II}
\author{Bernardo Mondragón Brozon}
\date{Enero de 2015}

\everymath{\displaystyle}

\DeclareMathOperator{\sech}{sech}
\DeclareMathOperator{\sen}{sen}
\DeclareMathOperator{\senh}{senh}
\DeclareMathOperator{\csch}{csch}

% Para cambiar de color el texto. La lista de colores se encuentra en la siguiente liga: "https://www.sharelatex.com/learn/Using_colours_in_LaTeX". Ejemplos en la liguiente liga: "https://www.sharelatex.com/project/59a64bd5cb832f0ec46b226b".
\newcommand{\Col}{\color{ProcessBlue}} 

% Integral
\newcommand{\xinteg}[4]{\int_{#1}^{#2} \! {#3} \, \mathrm{#4}}

% Integral con argumentos opcionales
\newcommandtwoopt{\integ}[4][][]{\int_{#1}^{#2} \! {#3} \, \mathrm{#4}}

% Derivada
\newcommand{\derivate}[2]{\frac{\mathrm{d}}{\mathrm{d}#1} \left(  {#2}  \right)  }

% Derivada sin parentesis
\newcommand{\der}[1]{\frac{\mathrm{d}}{\mathrm{d}#1}}

% Enesima derivada 
\newcommand{\derivada}[2]{\frac{\mathrm{d}^{#1}}{\mathrm{d}#2}}

% Limite
\newcommand{\limit}[2]{\lim_{#1\to #2}}

% Limite a infinito
\newcommand{\limf}[1]{\lim_{#1\to\infty}}

% Suma
\newcommand{\suma}[3]{\sum_{#1}^{#2}{#3}}

% Algo perteneciente a algo
\newcommand{\paratodoxen}[2]{\quad \forall {#1}\in\mathbb{#2}}

% Permutaciones
\newcommand*{\Perm}[2]{{}^{#1}\!P_{#2}}

% Combinaciones
\newcommand*{\Comb}[2]{{}^{#1}C_{#2}}

\iffalse
\begin{equation*}
\begin{split}
     \mbox{Empezar aqui}  & \mbox{Continuar aqui}  \\
 & \mbox{Lo que sigue abajo} \\
 & \mbox{Lo que sigue mas abajo} \}
\end{split}
\end{equation*}

\begin{equation*}
\begin{split}
 a^xa^y & =\exp(x\ln(a))\cdot\exp(y\ln(a)) \\
 		& =\exp(x\ln(a)+y\ln(a))  \\
 		& =\exp(\ln(a)(x+y)) \}
\end{split}
\end{equation*}

\fi

\begin{document}

\maketitle

\section{\Col El teorema fundamental del cálculo}

\subsection{\Col Area debajo de una gráfica}

\begin{itemize}
\item[\Col •] Se quiere hallar el área debajo de la gráfica de una función:

$$ incluir grafico $$

\item[\Col•] \textbf{Definición:} Sea $f$ una función continua en $[a,b]$ no negativa y sea $\mathbb{A}:[a,b]\longrightarrow\mathbb{R}$ como sigue:

$$ \mathbb{A}(x) =
\left\{
 \begin{array}{ll}
  0  & \mbox{si } x=a \\
  \\ \mbox{área bajo $f$} & \mbox{si } x\in [a,b]
 \end{array}
\right.$$
 
$\mathbb{A}(b)$ denota el área acumulada debajo de la gráfica de $f$ en el intervalo [a,b]. 

\item[\Col •] $\mathbb{A}:[a,b]\longrightarrow\mathbb{R}$ satisface que:

\begin{enumerate}
\item[a)] Por definición, $\mathbb{A}(a)=0$
\item[b)] Si $c\in [a,b]$, entonces $\mathbb{A}(a,b)=\mathbb{A}(a,c)+\mathbb{A}(c,b)$
\item[c)] Si $m$ es mínimo y $M$ es el máximo de la función en el intervalo $[a,b]$, entonces $m\leq f(x)\leq M$ y $m(b-a)\leq \mathbb{A}(b)\leq M(b-a)$ 
\end{enumerate} 
 
$$ incluir grafico $$

\item[\Col •] La notación integral: 

$$ \mathbb{A}(x)=\int_{a}^{x} \! {f(t)} \, \mathrm{dt} =
\left\{
 \begin{array}{ll}
  0  & \mbox{si } x=a \\
  \\ \mbox{área bajo $f$} & \mbox{si } x\in [a,b]
 \end{array}
\right.$$

\end{itemize}

\subsection{\Col El teorema del valor medio para integrales (TVMI)}

\begin{itemize}

\item[\Col •] \textbf{Teorema:} El teorema del valor medio para integrales ($TVMI$) estable que se si $f:[a,b]\longrightarrow\mathbb{R}$ es continua, entonces existe $c\in [a,b]$ tal que $\int_{a}^{b} \! {f(t)} \, \mathrm{dt}=f(c)(b-a)$.

\textbf{Demostración:}

Sea $m$ y $M$ tal que $m\leq f(x)\leq M$ $\forall x\in [a,b]$, entonces
$$ m(b-a)\leq \int_{a}^{b} \! {f(t)} \, \mathrm{dt}\leq M(b-a) $$ 
$$ \Rightarrow m\leq \frac{\int_{a}^{b} \! {f(t)} \, \mathrm{dt}}{b-a}\leq M $$

El teorema del Bolzano establece que una función continua "no se salta valores", entonces

$$ f(s)=m\leq \frac{\int_{a}^{b} \! {f(t)} \, \mathrm{dt}}{b-a}\leq M=f(t) \quad \mbox{para algún} \quad s,t\in [a,b] $$
$$ \Rightarrow f(c)=\frac{\int_{a}^{b} \! {f(t)} \, \mathrm{dt}}{b-a} \quad \mbox{para algún} \quad c\in [a,b] $$

es decir, $f(c)=\frac{\int_{a}^{b} \! {f(t)} \, \mathrm{dt}}{b-a}$ es un valor que alcanza la función, pues $f(c)=\frac{\int_{a}^{b} \! {f(t)} \, \mathrm{dt}}{b-a}$ está entre el mínimo y el máximo.

$$\Rightarrow f(c)(b-a)=\int_{a}^{b} \! {f(t)} \, \mathrm{dt} \quad \mbox{para algún} \quad c\in [a,b] $$

\begin{flushright}
$\blacksquare$
\end{flushright}

\end{itemize}

\subsection{\Col El Teorema fundamental del cálculo (TFC)}

\begin{itemize}

\item[\Col •] \textbf{Teorema:} El teorema fundamental del cálculo parte 1 ($TFC \#1$) establece que si $f:[a,b]\longrightarrow\mathbb{R}$ es una función continua, entonces $\mathbb{A}(x)=\int_{a}^{x} \! {f(t)} \, \mathrm{dt}$ es una función continua y diferenciable tal que

$$ \derivate{x}{\mathbb{A}(x)}=\derivate{x}{\int_{a}^{x} \! {f(t)} \, \mathrm{dt}}=f(x) $$ 

\textbf{Demostración:}

\textbf{Caso 1.} ($h>0$)

$$ \frac{d}{dx}\mathbb{A}(x)=\lim_{h\to 0}\frac{\mathbb{A}(x+h)-\mathbb{A}(x)}{h}= \lim_{h\to 0}\frac{  \int_{a}^{x+h} \! {f(t)} \, \mathrm{dt} - \int_{a}^{x} \! {f(t)} \, \mathrm{dt} }{h}$$
$$= \lim_{h\to 0}\frac{  \int_{a}^{x} \! {f(t)} \, \mathrm{dt} + \int_{x}^{x+h} \! {f(t)} \, \mathrm{dt} - \int_{a}^{x} \! {f(t)} \, \mathrm{dt} }{h}=\lim_{h\to 0} \frac{\int_{x}^{x+h} \! {f(t)} \, \mathrm{dt}}{h} $$

Por el teorema del valor medio para integrales, existe $c\in [x,x+h]$ tal que 

$$\lim_{h\to 0} \frac{\int_{x}^{x+h} \! {f(t)} \, \mathrm{dt}}{h}=\lim_{h\to 0}\frac{f(c)(x+h-x)}{h}=\lim_{h\to 0}f(c)$$

si $h>0$, entonces $x<x+h$ pero $c\in [x,x+h]$, de manera que si $h\longrightarrow 0^+$, entonces $c\longrightarrow x^+$, entonces $\lim_{h\to 0^+}f(c)=f(x)$, entonces $D^+(\mathbb{A}(x))=f(x)$

\textbf{Caso 2.} ($h<0$)

$$ \frac{d}{dx}\mathbb{A}(x)=\lim_{h\to 0}\frac{\mathbb{A}(x+h)-\mathbb{A}(x)}{h}= \lim_{h\to 0}\frac{  \int_{a}^{x+h} \! {f(t)} \, \mathrm{dt} - \int_{a}^{x} \! {f(t)} \, \mathrm{dt} }{h}$$
$$= \lim_{h\to 0}\frac{  \int_{a}^{x+h} \! {f(t)} \, \mathrm{dt} -\left( \int_{a}^{x+h} \! {f(t)} \, \mathrm{dt} + \int_{x+h}^{x} \! {f(t)} \, \mathrm{dt} \right)}{h}=\lim_{h\to 0} \frac{-\int_{x+h}^{x} \! {f(t)} \, \mathrm{dt}}{h} $$

Por el teorema del valor medio para integrales, existe $c\in [x+h,x]$ tal que 

$$\lim_{h\to 0} \frac{-\int_{x+h}^{x} \! {f(t)} \, \mathrm{dt}}{h}=\lim_{h\to 0}\frac{-f(c)(x-(x+h))}{h}=\lim_{h\to 0}f(c)$$

si $h<0$, entonces $x+h<x$ pero $c\in [x,x+h]$, de manera que si $h\longrightarrow 0^-$, entonces $c\longrightarrow x^-$, entonces $\lim_{h\to 0^-}f(c)=f(x)$, entonces $D^-(\mathbb{A}(x))=f(x)$

Por el caso 1 y por el caso 2, se tiene que $D^+(\mathbb{A}(x))=f(x)$ y $D^-(\mathbb{A}(x))=f(x)$, entonces $\mathbb{A}'(x)=f(x)$. \\

	Por el teorema fundamental del cálculo parte 1, vemos que el problema de hallar la función de área se convierte en hallar una primitiva de la función bajo la cual se quiere encontrar el área. 
	
\begin{flushright}
$\blacksquare$
\end{flushright}	

\item[\Col •] \textbf{Teorema:} El teorema fundamental del cálculo parte 2 ($TFC \#2$) establece que si $f:[a,b]\longrightarrow\mathbb{R}$ es una función continua, y $\derivate{x}{F(x)}=f(x)$, $\forall x\in [a,b]$, entonces 
$$ \int_{a}^{b} \! {f(t)} \, \mathrm{dt}=F(b)-F(a)=F(t) \Big|_a^b $$ 

\textbf{Demostración:}

Sea $G(x)=\int_{a}^{x} \! {f(t)} \, \mathrm{dt}$ y sea $F(x)$ tal que $F'(x)=f(x)$, antiderivadas de $f(x)$ (dos funciones cuyas derivadas son $f(x)$), entonces $F(x)$ y $G(x)$ difieren por una contante $C$:

$$ F(x)=G(x)+C \quad \forall x\in [a,b] $$

Ahora hacemos:

$$\int_{a}^{b} \! {f(t)} \, \mathrm{dt}=\int_{a}^{b} \! {f(t)} \, \mathrm{dt}-0=\int_{a}^{b} \! {f(t)} \, \mathrm{dt}- \int_{a}^{a} \! {f(t)} \, \mathrm{dt}$$
$$=G(b)-G(a)=G(b)+c-(G(a)+c)=F(b)-F(a)$$

Por lo tanto, 

$$ \int_{a}^{b} \! {f(t)} \, \mathrm{dt}=F(b)-F(a) $$

\begin{flushright}
$\blacksquare$
\end{flushright}

\item[\Col •] Convención: si $b\leq a$, entonces $\int_{a}^{b} \! {f(t)} \, \mathrm{dt}=-\int_{b}^{a} \! {f(t)} \, \mathrm{dt}$

\item[\Col •] Se tiene que interpretar correctamente la definición de valor absoluto:  sean $a$ y $b$ tal que $a\leq b$. Si se tiene $\int_{a}^{b} \! {|t|} \, \mathrm{dt}$, entonces hay tres casos:

\textbf{Caso 1.} $b\leq 0$, entonces $\int_{a}^{b} \! {|t|} \, \mathrm{dt}=\int_{a}^{b} \! {-t} \, \mathrm{dt}=-\int_{a}^{b} \! {t} \, \mathrm{dt}=-\frac{t^2}{2}\Big|_a^b=\frac{a^2-b^2}{2}>0$.

\textbf{Caso 2.} $a\leq 0\leq b$, entonces $\int_{a}^{b} \! {|t|} \, \mathrm{dt}=\int_{a}^{0} \! {-t} \, \mathrm{dt}+\int_{0}^{b} \! {t} \, \mathrm{dt}=-\int_{a}^{0} \! {t} \, \mathrm{dt}+\int_{0}^{b} \! {t} \, \mathrm{dt}=-\frac{t^2}{2}\Big|_a^0+\frac{t^2}{2}\Big|_0^b=\frac{a^2+b^2}{2}>0$.

\textbf{Caso 3.} $a\geq 0$, entonces $\int_{a}^{b} \! {|t|} \, \mathrm{dt}=\int_{a}^{b} \! {t} \, \mathrm{dt}=\frac{t^2}{2}\Big|_a^b=\frac{b^2-a^2}{2}>0$.

$$ incluirgrafico $$

\end{itemize}

\newpage

\section{\Col El logaritmo natural}

\subsection{\Col Propiedades del logaritmo natural}

\begin{itemize}
\item[\Col •] La regla del exponente dice que $\int_{}^{} \! {t^r} \, \mathrm{dt}=\frac{t^{r+1}}{r+1}+C$, $\forall r\neq -1$. ¿Qué pasa si $r=-1$? No se puede aplicar la regla del exponente, entonces buscamos una función diferenciable $L(x)$ tal que 

$$ L'(x)=x^{-1}=\frac{1}{x} \quad \forall x\neq 0 $$

\item[\Col •] \textbf{Definición:} El logaritmo natural es una función $ln:\mathbb{R^+}\longrightarrow\mathbb{R}$ definida como sigue:
$$ \ln(x)=\xinteg{1}{x}{\frac{1}{t}}{dt}=\xinteg{1}{x}{t^{-1}}{dt}$$ 

El logaritmo natural es el área debajo de la gráfica de la función $f(x)=\frac{1}{x}$ en el intervalo $[1,a]$:

$$ incluirgrafico $$ 

\item[\Col •] \textbf{Observacion:} Por el TFC \#1, $\derivate{x}{\ln(x)}=   \derivate{x}{\xinteg{1}{x}{\frac{1}{t}}{dt}}=\frac{1}{x}$.

\item[\Col •] Sea $F(t)=\ln(-t)+C$ y $G(t)=\ln(t)+C$, entonces $\frac{d}{dt}F(t)=\frac{1}{t}$ y $\frac{d}{dt}G(f)=\frac{1}{t}$, entonces se sigue que 

$$ \xinteg{}{}{\frac{1}{t}}{dt}= \left\{
 \begin{array}{ll}
  \ln(-t)+C  & \mbox{si } t<0 \\
  \\ \ln(t)+C & \mbox{si } t>0
 \end{array}
\right.$$

Por lo tanto $\xinteg{}{}{\frac{1}{t}}{dt}=\ln|t|+C$ si $t\neq 0$. En general, $\xinteg{}{}{\frac{u'}{u}}{du}=\ln|u|+C$ si $u\neq 0$.

\item[\Col •] \textbf{Proposición:} Propiedades del logaritmo natural. $\forall a,b>0$ y $\forall r\in\mathbb{Q}$ ocurre que:
\begin{enumerate}
\item[a)] $\ln(ab)=\ln(a)+\ln(b)$
\item[b)] $\ln(a^{-1})=\ln\left(\frac{1}{a}\right)=-\ln(a)$
\item[c)] $\ln\left(\frac{a}{b}\right)=\ln(a)-\ln(b)$
\item[d)] $\ln(a^r)=r\ln(a)$
\end{enumerate}

\textbf{Demostración:}

\begin{enumerate}
\item[a)] \textbf{P.D.} $\ln(ab)=\ln(a)+\ln(b)$

Sea $f(b)=\ln(ab)\Rightarrow\frac{d}{db}f(b)=\frac{1}{b}$. Sea $g(b)=\ln(b)\Rightarrow\frac{d}{db}g(b)=\frac{1}{b}$, entonces $f(b)$ y $g(b)$ tienen la misma derivada, por lo tanto, difieren por una contante $C$, $\forall b\in (0,\infty)$, si y sólo si $F(b)=G(b)+C$, $\forall b\in (0,\infty)$. Si $b=1\Rightarrow F(1)=G(1)+C=\ln(1)+C=0+C=C \Leftrightarrow F(1)=C$, pero $F(1)=\ln(a(1))=\ln(a)\Leftrightarrow F(1)=C=\ln(a)\Leftrightarrow C=\ln(a)\Leftrightarrow F(b)=\ln(b)+\ln(a)=\ln(a)+\ln(b) \therefore \ln(ab)=\ln(a)+\ln(b) \quad \forall b\in (0,\infty)$.

\item[b)] \textbf{P.D.} $\ln(a^{-1})=\ln\left(\frac{1}{a}\right)=-\ln(a)$

Si $b>0\Leftrightarrow\frac{1}{b}>0$. $1=\frac{b}{b}\Leftrightarrow 1=bb^{-1}\Leftrightarrow\ln(1)=\ln(bb^{-1})\Leftrightarrow 0=\ln(bb^{-1})\Leftrightarrow 0=\ln(b)+ln(b^{-1})\Leftrightarrow \ln(b^-1)=-\ln(b)$. 

\item[c)] \textbf{P.D.} $\ln\left(\frac{a}{b}\right)=\ln(a)-\ln(b)$

$\ln\left(\frac{a}{b}\right)=\ln(ab^{-1})=\ln(a)+\ln(b^{-1})=\ln(a)-\ln(b)$.

\item[d)] \textbf{P.D.} $\ln(a^r)=r\ln(a)$

Sea $F(a)=\ln(a^{r})\Rightarrow \frac{d}{da}F(a)=\frac{r}{a}$. Sea $G(a)=r\ln(a)\Rightarrow\frac{d}{da}G(a)=\frac{r}{a}$, entonces $F(a)$ y $G(a)$ tienen la misma derivada, por lo tanto, difieren por una constante $C$, $\forall a\in (0,\infty)$, si y sólo si $F(a)=G(a)+C$, $\forall a\in (0,\infty)$.

 Si $a=1 \Rightarrow F(1)=G(1)+C=r\ln(1)+C=0+C=C\Leftrightarrow F(1)=C$, pero $F(1)=\ln(1^{r})=\ln(1)=0\Leftrightarrow C=0\Leftrightarrow F(a)=G(a)\Leftrightarrow r\ln(a)=\ln(a^r)\Leftrightarrow \ln(a^r)=r\ln(a)$. 

\begin{flushright}
$\blacksquare$
\end{flushright}

\end{enumerate}

\item[\Col •] En las demostraciones anteriores se usó el hecho de que dos funciones que tiene la misma derivada difieren por una constante.

\end{itemize}

\subsection{\Col Límites relevantes del logaritmo natural}

\begin{itemize}

\item[\Col •] Los siguientes teoremas serán de gran utilidad para conocer el comportamiento del logaritmo natural y su gráfica.

\item[\Col •] \textbf{Teorema:} $\lim_{x\to \infty}\ln(x)=\infty$

\textbf{Demostración:}

Se prueba por inducción que $\sum_{i=1}^{2^n}\frac{1}{i}\geq 1+\frac{n}{2}$. Además, $\lim_{n\to \infty}\left(a+\frac{n}{2}\right)=\infty$, entonces $\lim_{n\to \infty}\sum_{i=1}^{2^n}\frac{1}{i}=\sum_{i=1}^{2^\infty}\frac{1}{i}=\sum_{i=1}^{\infty}\frac{1}{i}=\infty$. Luego, se observa que, $\forall n>1$, $(1)\frac{1}{2}+(1)\frac{1}{3}+\cdots+(1)\frac{1}{n}<\ln(n)$, tal y como se muestra en la siguiente gráfica:

$$ incluirgrafico $$  

Entonces $\forall n>1$, $(1)\frac{1}{2}+(1)\frac{1}{3}+\cdots+(1)\frac{1}{n}<\ln(n)$, así pues $\frac{1}{2}+\frac{1}{3}+\cdots+\frac{1}{n}<\ln(n)$, luego $\left(1+\frac{1}{2}+\frac{1}{3}+\cdots+\frac{1}{n}\right)-1<\ln(n)$, por tanto $\lim_{n\to \infty}\left(1+\frac{1}{2}+\frac{1}{3}+\cdots+\frac{1}{n}\right)-1 <\lim_{n\to\infty}\ln(n)$, se sigue que $\lim_{n\to \infty}\left(\sum_{i=1}^{2^n}\frac{1}{i}-1\right)<\lim_{n\to\infty}\ln(n)$, entonces $\infty=\lim_{n\to \infty}\left(\sum_{i=1}^{2^n}\frac{1}{i}-1\right)<\lim_{n\to\infty}\ln(n)$, por lo tanto $\lim_{x\to\infty}\ln(x)=\infty$.

\begin{flushright}
$\blacksquare$
\end{flushright}

\item[\Col •] \textbf{Teorema:} $\lim_{x\to 0^+}\ln(x)=-\infty$

\textbf{Demostración:}

Sea $t=\frac{1}{x}$. Si $x\to 0^+$, entonces $t\to\infty$, además, si $t\to\infty$, entonces $x\to 0^+$, por lo tanto, $t\to\infty$, si y sólo si $x\to 0^+$. De manera que

$$\lim_{x\to 0^+}\ln(x)=\lim_{t\to\infty}\ln\left(\frac{1}{t}\right)=\lim_{t\to\infty}\ln(t^{-1})=\lim_{t\to\infty}-\ln(t)=-\lim_{t\to\infty}\ln(t)=-\infty$$

\begin{flushright}
$\blacksquare$
\end{flushright}

\item[\Col •] \textbf{Teorema:} $\lim_{x\to\infty}\frac{\ln(x)}{x}=0$

\textbf{Demostración:} 

Si $1\leq t$, entonces $t\leq t^2$, entonces $\sqrt{t}\leq \sqrt{t^2}$, entonces $\sqrt{t}\leq t$, entonces $\frac{1}{\sqrt{t}}\geq\frac{1}{t}$, tal y como se muestra en la siguiente gráfica:

$$ incluirgrafico$$

Por lo tanto, si $x\geq 1$, entonces $ln(x)=\xinteg{1}{x}{\frac{1}{t}}{dt}\leq\xinteg{1}{x}{\frac{1}{\sqrt{t}}}{dt}=2\sqrt{x}-2<2\sqrt{x}$, entonces $\ln(x)<2\sqrt{x}$. Por lo tanto, $0<\ln(x)<2\sqrt{x} \quad \forall x\geq 1$, entonces $0<\frac{\ln(x)}{x}<\frac{2\sqrt{x}}{x}$, entonces $\lim_{x\to\infty}0<\lim_{x\to\infty}\frac{\ln(x)}{x}<\lim_{x\to\infty}\frac{2\sqrt{x}}{x}$, entonces $0<\lim_{x\to\infty}\frac{\ln(x)}{x}<0$, entonces $\lim_{x\to\infty}\frac{\ln(x)}{x}=0$

\begin{flushright}
$\blacksquare$
\end{flushright}

\item[\Col •] \textbf{Teorema:} Si $a$ y $b$ con constantes, entonces $\lim_{x\to\infty}\frac{\ln(x)}{ax+b}=0$

\textbf{Demostración:}

$$ \lim_{x\to\infty}\frac{\ln(x)}{ax+b}=\lim_{x\to\infty}\frac{\frac{\ln(x)}{x}}{\frac{ax+b}{x}}=\frac{\lim_{x\to\infty}\frac{\ln(x)}{x}}{\lim_{x\to\infty}\frac{ax+b}{x}}=\frac{0}{a+0}=0 $$

\begin{flushright}
$\blacksquare$
\end{flushright}

\end{itemize}

\subsection{\Col La gráfica de la función logaritmo natural}

\begin{itemize}

\item[\Col •] A continuación se trazará con todo detalle la gráfica de la función logaritmo natural enunciando todas sus propiedades: 0) Dominio, imagen y puntos por los que pasa la función; 1) Primera derivada, intervalos de crecimiento y decrecimiento, y puntos críticos estacionarios o puntos singulares; 2) Segunda derivada, intervalos de concavidad positiva y negativa, y puntos de inflexión; 3) Límites relevantes y comportamiento asintótico. 

\begin{enumerate}
\item[0)] $Dom(\ln)=(0,\infty)$, $Im(\ln)=(-\infty,\infty)$ y Pasa por los puntos $(1,0)$ y $(e,1)$.
\item[1)] $\ln'(x)=\frac{1}{x}>0 \quad \forall x\in (0,\infty)$. Por lo tanto:
\begin{enumerate}
\item Intervalo de crecimiento$=(0,\infty)$
\item Intervalo de decrecimiento$=\varnothing$
\item No existen puntos críticos estacionarios
\end{enumerate} 
\item[2)] $\ln''(x)=-\frac{1}{x^2}<0 \quad \forall x\in (0,\infty)$. Por lo tanto:
\begin{enumerate}
\item Intervalo de concavidad positiva$=\varnothing$
\item Intervalo de concavidad negativa$=(0,\infty)$
\item No existen puntos de inflexión
\end{enumerate}
\item[3)] $\lim_{x\to\infty}\ln(x)=\infty$, $\lim_{x\to 0^+}\ln(x)=-\infty$, $\lim_{x\to\infty}\frac{\ln(x)}{x}=0$. Por lo tanto hay una asíntota vertical en $x=0$.
\end{enumerate}

\item[\Col •] Dado lo anterior, la gráfica de la función logaritmo natural tiene el siguiente aspecto:

$$ incluirgrafico $$

\end{itemize}

\subsection{\Col Algunos límites importantes}

\begin{itemize}
\item[\Col •] \textbf{Teorema:} $\lim_{x\to 0^+}\frac{\ln(x)}{x}=-\infty$

\textbf{Demostración:} 

$$ \lim_{x\to 0^+}\frac{\ln(x)}{x}=\lim_{x\to 0^+}\ln(x)\left(\frac{1}{x}\right)="\infty\cdot-\infty"=-\infty $$

\begin{flushright}
$\blacksquare$
\end{flushright}

\item[\Col •] \textbf{Teorema:} $\lim_{x\to 0^+}x\ln(x)=0$

\textbf{Demostración:} 

Sea $t=\frac{1}{x}$. Si $x\to 0^+$, entonces $t\to\infty$, además, si $t\to\infty$, entonces $x\to 0^+$, por lo tanto, $t\to\infty$, si y sólo si $x\to 0^+$. De manera que

$$\lim_{x\to 0^+}x\ln(x)=\lim_{t\to\infty}\frac{\ln\left(\frac{1}{t}\right)}{t}=\lim_{t\to\infty}-\frac{\ln(t)}{t}=-0=0$$

\begin{flushright}
$\blacksquare$
\end{flushright}

\item[\Col •] \textbf{Teorema:} $\lim_{x\to\infty}\frac{\ln(1+x^2)}{x}=0$

\textbf{Demostración:}

Si $a<b<c$, entonces $\ln(a)<\ln(b)<\ln(c)$, por lo que $\ln(x^2)<\ln(1+x^2)<\ln(x^2+x^2)$, de manera que $2\ln(x)<\ln(1+x^2)<\ln(2x^2)$, así pues $2\ln(x)<\ln(1+x^2)<\ln(2)+\ln(x^2)$, entonces $\frac{2\ln(x)}{x}<\frac{\ln(1+x^2)}{x}<\frac{\ln(2)+\ln(x^2)}{x}$, se sigue que $\lim_{x\to\infty}\frac{2\ln(x)}{x}<\lim_{x\to\infty}\frac{\ln(1+x^2)}{x}<\lim_{x\to\infty}\frac{\ln(2)+\ln(x^2)}{x}$, de ahí que $0<\lim_{x\to\infty}\frac{\ln(1+x^2)}{x}<0$, por lo tanto $\lim_{x\to\infty}\frac{\ln(1+x^2)}{x}=0$.

\begin{flushright}
$\blacksquare$
\end{flushright}

\end{itemize}

\subsection{\Col Primitivas trigonométricas}

\begin{itemize}
\item[\Col •] Con lo anterior, ahora podemos hallar las primivitvas de las principales funciones trigonométricas. 

\item[\Col •] Recordar que $\xinteg{}{}{\frac{u'}{u}}{du}=\ln|u|+C$.

\item[\Col •] Las primitivas de las principales funciones trigonométricas son las siguientes:

\begin{enumerate}
\item $\xinteg{}{}{\sin(t)}{dt}=-\cos(t)+C$
\item $\xinteg{}{}{\cos(t)}{dt}=\sin(t)+C$
\item $\xinteg{}{}{\tan(t)}{dt}=\xinteg{}{}{\frac{\sin(t)}{\cos(t)}}{dt}=-\xinteg{}{}{\frac{-\sin(t)}{\cos(t)}}{dt}=-\xinteg{}{}{\frac{\cos'(t)}{\cos(t)}}{dt}=-\ln|\cos(t)|+C$
\item $\xinteg{}{}{\cot(t)}{dt}=\xinteg{}{}{\frac{\cos(t)}{\sin(t)}}{dt}=\xinteg{}{}{\frac{\sin'(t)}{\cos(t)}}{dt}=\ln|\sin(t)|+C$
\item $\xinteg{}{}{\sec(t)}{dt}=\xinteg{}{}{\sec(t)\\frac{\sec(t)+\tan(t)}{\sec(t)+\tan(t)}}{dt}=\xinteg{}{}{\frac{\sec^2(t)+\sec(t)\tan(t)}{\sec(t)+\tan(t)}}{dt}$

$=\xinteg{}{}{\frac{(\sec(t)+\tan(t))'}{\sec(t)+\tan(t)}}{dt}=\ln|\sec(t)+\tan(t)|+C$
\item $\xinteg{}{}{\csc(t)}{dt}=\xinteg{}{}{\csc(t)\frac{\cot(t)-\csc(t)}{\cot(t)-\csc(t)}}{dt}=\xinteg{}{}{\frac{\csc(t)\cot(t)-\csc^2(t)}{\cot(t)-\csc(t)}}{dt}$

$=\xinteg{}{}{\frac{(\cot(t)-\csc(t))'}{\cot(t)-\csc(t)}}{dt}=\ln|\cot(t)-\csc(t)|+C$
\end{enumerate}

\end{itemize}

\subsection{\Col Diferenciación logarítmica}

\begin{itemize}
\item[\Col •] Cuando se tiene una función muy "exótica", en algunas ocaciones, es más conveniente utilizar la diferenciación logarítmica.

\item[\Col •] \textbf{Teorema:} Sea $y=f(x)$, entonces $y'=y(\ln(f(x)))'$.

\textbf{Demostración:}
  
Si $y=f(x)$, entonces $\ln(y)=\ln(f(x))$, de manera que $\frac{d}{dx}\ln(y)=\frac{d}{dx}\ln(f(x))$, así pues, $\frac{y'}{y}=(\ln(f(x)))'$, por lo tanto $y'=y(\ln(f(x)))'$.  
  
\begin{flushright}
$\blacksquare$
\end{flushright}  
  
\end{itemize}

\newpage

\section{\Col La exponencial} 

\subsection{\Col Propiedades de la exponencial}

\begin{itemize}

\item[\Col •] Trazando la gráfica de la función del logaritmo natural, se probó que $\ln:(0,\infty)\longrightarrow\mathbb{R}$ es una función inyectiva y suprayectiva, y por lo tanto, $\ln(x)$ es una función biyectiva. Podemos concluir que $\ln(x)$ tiene función inversa, es decir, $\ln(x)$ es invertible.

\item[\Col •] \textbf{Definición:} La exponencial es una función $\exp:\mathbb{R}\longrightarrow (0,\infty)$ y se define como la inversa bajo composición de la función del logaritmo natural. De manera que 

$$ y=\ln(x), \quad \mbox{si y sólo si} \quad x=\exp(y)$$ 

\item[\Col •] Si obtenemos la función exponencial a partir de la inversa de la función $\ln(x)$, habría que hacer una reflexión de $\ln(x)$ con respecto al origen tal y como se muestra en la siguiente figura:

$$ incluirgrafico $$ 

\item[\Col •] \textbf{Teorema:} Si $f(x)$ es una función estrictamente creciente tal que existe su inversa, entonces $f^{-1}(x)$ es una función estrictamente creciente, es decir, la inversa de una función creciente es una funcion creciente.

\textbf{Demostración:} Sea $a<b$. Supongamos que $f^{-1}(a)\geq f^{-1}(b)$, entonces, como $f(x)$ es una función creciente, $f(f^{-1}(a))\geq f(f^{-1}(b))$, entonces $a\geq b$, lo cual es una contradicción, por lo tanto, por reducción a lo absurdo, tiene que suceder que $f^{-1}(a)<f^{-1}(b)$.

\begin{flushright}
$\blacksquare$
\end{flushright} 

\item[\Col •] \textbf{Proposición:} Propiedades de la exponencial. $\forall a,b\in\mathbb{R}$ y $\forall r\in\mathbb{Q}$ ocurre que:
\begin{enumerate}
\item[a)] $\exp(a+b)=\exp(a)\exp(b)$
\item[b)] $\exp(a-b)=\frac{\exp(a)}{\exp(b)}$
\item[c)] $\exp(-b)=\frac{1}{\exp(b)}=(\exp(b))^{-1}$
\item[d)] $\exp(ra)=(\exp(a))^r$
\end{enumerate}

\textbf{Demostración:}

\begin{enumerate}
\item[a)] \textbf{P.D.} $\exp(a+b)=\exp(a)\exp(b)$

Sea $x=\exp(a)$ y $y=\exp(b)$, entonces $a=\ln(x)$ y $b=\ln(y)$, luego $\ln(xy)=\ln(x)+\ln(y)=a+b$, 
así pues, $\ln(xy)=a+b$, de manera que $\exp(\ln(xy))=\exp(a+b)$, se sigue que $xy=\exp(a+b)$, por lo tanto $\exp(a+b)=\exp(a)\exp(b)$.

\item[b)] \textbf{P.D.} $\exp(a-b)=\frac{\exp(a)}{\exp(b)}$

Sea $x=\exp(a)$ y $y=\exp(b)$, entonces $a=\ln(x)$ y $b=\ln(y)$, luego $\ln\left(\frac{x}{y}\right)=\ln(x)-\ln(y)=a-b$, así pues $\ln\left(\frac{x}{y}\right)=a-b$, de manera que $\exp\left(\ln\left(\frac{x}{y}\right)\right)=\exp(a-b)$, se sique que $\frac{x}{y}=\exp(a-b)$, por lo tanto $\exp(a-b)=\frac{\exp(a)}{\exp(b)}$.

\item[c)] \textbf{P.D.} $\exp(-a)=\frac{1}{\exp(a)}=(\exp(a))^{-1}$

Sea $x=\exp(a)$, entonces $a=\ln(x)$, luego $\ln\left(\frac{1}{x}\right)=\ln(x^{-1})=-\ln(x)=-a$, así pues $\ln\left(\frac{1}{x}\right)=-a$, de manera que $\exp\left(\ln\left(\frac{1}{x}\right)\right)=\exp(-a)$, se sique que $\frac{1}{x}=\exp(-a)$, por lo tanto $\exp(-a)=\frac{1}{\exp(a)}=(\exp(a))^{-1}$.

\item[d)] \textbf{P.D.} $\exp(ra)=(\exp(a))^r$

Sea $x=\exp(a)$, entonces $a=\ln(x)$, luego $\ln(x^r)=r\ln(x)$, así pues $\exp(\ln(x^r))=\exp(r\ln(x))$, de manera que $x^r=\exp(r\ln(x))$, se sigue que $\exp(ra)=(\exp(a))^{r}$.

\begin{flushright}
$\blacksquare$
\end{flushright} 

\end{enumerate}

\end{itemize}

\section{\Col Límites relevantes de la exponencial}

\begin{itemize}
\item[\Col •] \textbf{Teorema:} $\ln(x)\leq x-1 \quad \forall x\geq 1$. \textbf{(No está tan clara la demostración de este teorema)}

\textbf{Demostración:} Por el teorema del valor medio para integrales, se tiene que existe algúna $c\in [1,x]$ tal que

$$ \ln(x)=\xinteg{1}{x}{\frac{1}{t}}{dt}=f(c)(x-1)=\frac{1}{c}(x-1) $$

Por lo tanto, si $c=1$, entonces $\ln(x)=x-1$, pero si $c=x$, entonces $\ln(x)=\frac{x-1}{x}$. Además, si $x\geq 1$ se tiene que 

$$ (x-1)^2=x^2-2x-1\geq 0 \Rightarrow x^2-2x\geq 1 \Rightarrow x^2-x\geq 1+x $$
$$ \Rightarrow x(x-1)\geq 1+x \Rightarrow x-1\geq \frac{1+x}{x} \geq \frac{x-1}{x} $$

Se sigue que $\frac{x-1}{x}\leq\ln(x)\leq x-1$, en particular $\ln(x)\leq x-1$.

\begin{flushright}
$\blacksquare$
\end{flushright} 

\item[\Col •] \textbf{Teorema:} $\lim_{x\to \infty}\exp(x)=\infty$

\textbf{Demostración:}

Si $x\geq 1$, entonces $\ln(x)\leq x-1\leq x$, como la función esponencial es una función creciente, se tiene que $\exp(\ln(x))\leq\exp(x)$, se sigue que $x\leq\exp(x)$, de manera que
$$ \lim_{x\to\infty}x\leq\lim_{x\to\infty}\exp(x)\Rightarrow \infty\leq\lim_{x\to\infty}\exp(x) $$
$$ \therefore \lim_{x\to\infty}\exp(x)=\infty $$

\begin{flushright}
$\blacksquare$
\end{flushright} 

\item[\Col •] \textbf{Teorema:} $\lim_{x\to -\infty}\exp(x)=0$

\textbf{Demostración:} 

Sea $t=-x$. Si $x\to -\infty$, entonces $t\to\infty$, además, si $t\to\infty$, entonces $x\to-\infty$, por lo tanto, $x\to -\infty$, si y sólo si $t\to\infty$. De manera que

$$\lim_{x\to -\infty}\exp(x)=\lim_{t\to\infty}\exp(-t)=\lim_{t\to\infty}\frac{1}{\exp(t)}="0^+"=0$$

\begin{flushright}
$\blacksquare$
\end{flushright} 

\item[\Col •] \textbf{Teorema:} $\lim_{x\to\infty}\frac{x}{\exp(x)}=0$

\textbf{Demostración:} 

Sea $t=\exp(x)$. Si $x\to\infty$, entonces $t\to\infty$, además, si $t\to\infty$, entonces $x\to\infty$, por lo tanto, $x\to\infty$, si y sólo si $t\to\infty$. De manera que

$$ \lim_{x\to\infty}\frac{x}{\exp(x)}=\lim_{t\to\infty}\frac{\ln(t)}{t}=0 $$

\begin{flushright}
$\blacksquare$
\end{flushright} 

\item[\Col •] \textbf{Teorema:} Si $a$ y $b$ son constantes, entonces $\lim_{x\to\infty}\frac{ax+b}{\exp(x)}=0$

\textbf{Demostración:} 

$$ \lim_{x\to\infty}\frac{ax+b}{\exp(x)}=\lim_{x\to\infty}\left( \frac{ax}{\exp(x)}+\frac{b}{\exp(x)} \right)=\lim_{x\to\infty}\frac{ax}{\exp(x)}+\lim_{x\to\infty}\frac{b}{\exp(x)} $$
$$ =a\lim_{x\to\infty}\frac{x}{\exp(x)}+b\lim_{x\to\infty}\frac{1}{\exp(x)}=a(0)+b(0)=0 $$


\begin{flushright}
$\blacksquare$
\end{flushright} 

\end{itemize}


\subsection{\Col La gráfica de la función exponencial}

\begin{itemize}
\item[\Col •] Para trazar la gráfica de la funcion exponencial, primero será necesario probar que la función $\exp(x)$ es diferenciable.

\item[\Col •] \textbf{Teorema:} La función exponencial $\exp(x)$ es diferenciable en $\mathbb{R}$ y $\exp'(x)=\exp(x)$.

\textbf{Demostración:}

Dado que la función exponencial es la inversa de la función logaritmo, se tiene lo siguiente:

$$ \ln(\exp(x))=x \quad \forall x\in\mathbb{R} \Rightarrow \derivate{x}{\ln(\exp(x))}=\derivate{x}{x} $$
$$ \Rightarrow \ln'(\exp(x))(\exp'(x))=1 \Rightarrow \frac{1}{\exp(x)}(\exp'(x))=1 $$
$$ \Rightarrow \exp'(x)=\exp(x) $$

\begin{flushright}
$\blacksquare$
\end{flushright}

\item[\Col •] La prueba anterior implica que la función exponencial es diferenciable, pues sólo se puede usar la regla de la cadena para funciones ambas diferenciables. La demostración anterior utiliza el hecho de que la función exponencial es diferenciable, lo cual no se ha probado. A continuación se hará la prueba formal de diferenciabilidad de la función exponencial.

\item[\Col •] \textbf{Teorema:} La función exponencial $\exp(x)$ es diferenciable en $\mathbb{R}$ y $\exp'(x)=\exp(x)$.

\textbf{Demostración:}  

Para demostrar que la función exponencial es diferenciable, es necesario comprobar que exista el siguinete límite:

$$ \lim_{h\to 0}\left(\frac{\exp(x+h)-\exp(x)}{h}\right) $$

Pero 

$$ \lim_{h\to 0}\left(\frac{\exp(x+h)-\exp(x)}{h}\right)=\lim_{h\to 0}\left(\frac{\exp(x)\exp(h)-\exp(x)}{h}\right) $$
$$ =\lim_{h\to 0}\left(\frac{\exp(x)(\exp(h)-1)}{h}\right)=\exp(x)\lim_{h\to 0}\left(\frac{\exp(h)-1}{h}\right) $$

Sea $h=\ln(t)$. Si $h\to 0$, entonces $t\to 1$, además, si $t\to 1$, entonces $h\to 0$, por lo tanto, $h\to 0$, si y sólo si $t\to 1$. De manera que

$$ \exp(x)\lim_{h\to 0}\left(\frac{\exp(h)-1}{h}\right)=\exp(x)\lim_{t\to 1}\left(\frac{t-1}{\ln(t)}\right)=\exp(x)\lim_{t\to 1}\left(\frac{1}{\frac{\ln(t)}{t-1}}\right) $$
$$ =\exp(x)\lim_{t\to 1}\left(\frac{1}{\frac{\ln(t)-\ln(1)}{t-1}}\right) $$

Pero $ \lim_{t\to 1}\left(\frac{\ln(t)-\ln(1)}{t-1}\right) $ es lo mismo que la derivada de la función logaritmo evaluada en $1$ y $\ln'(1)=\frac{1}{1}=1$, por lo tanto

$$ \exp(x)\lim_{t\to 1}\left(\frac{1}{\frac{\ln(t)-\ln(1)}{t-1}}\right)=\exp(x) $$

\begin{flushright}
$\blacksquare$
\end{flushright}

\item[\Col •] A continuación se trazará con todo detalle la gráfica de la función exponencial enunciando todas sus propiedades: 0) Dominio, imagen y puntos por los que pasa la función; 1) Primera derivada, intervalos de crecimiento y decrecimiento, y puntos críticos estacionarios o puntos singulares; 2) Segunda derivada, intervalos de concavidad positiva y negativa, y puntos de inflexión; 3) Límites relevantes y comportamiento asintótico. 

\begin{enumerate}
\item[0)] $Dom(\exp)=(-\infty,\infty)$, $Im(\exp)=(0,\infty)$ y Pasa por los puntos $(0,1)$ y $(1,e)$.
\item[1)] $\exp'(x)=\exp(x)>0 \quad \forall x\in (-\infty,\infty)$. Por lo tanto:
\begin{enumerate}
\item Intervalo de crecimiento$=(-\infty,\infty)$
\item Intervalo de decrecimiento$=\varnothing$
\item No existen puntos críticos estacionarios
\end{enumerate} 
\item[2)] $\exp''(x)=\exp>0 \quad \forall x\in (-\infty,\infty)$. Por lo tanto:
\begin{enumerate}
\item Intervalo de concavidad positiva$=(-\infty,\infty)$
\item Intervalo de concavidad negativa$=\varnothing$
\item No existen puntos de inflexión
\end{enumerate}
\item[3)] $\lim_{x\to\infty}\exp(x)=\infty$, $\lim_{x\to -\infty}\exp(x)=0$, $\lim_{x\to\infty}\frac{ax+b}{\exp(x)}=0$. Por lo tanto hay una asíntota horizontal en $x=0$.
\end{enumerate}

\item[\Col •] Dado lo anterior, la gráfica de la función exponencial tiene el siguiente aspecto:

$$ incluir grafico $$

\end{itemize}

\subsection{\Col Más acerca de la exponencial}

\begin{itemize}
\item[\Col •] Dado que ya se demostró que la función esponencial $\exp(x)$ es una función diferenciable y $\exp'(x)=\exp(x)$, se puede hacer la siguiente observación, pues se sigue que 

$$ \integ[][]{\exp(x)}{dx}=\exp(x)+C $$

Lo cual da lugar al siguinete teorema.

\item[\Col •] \textbf{Teorema:} Si $f(x)$ es diferenciable, entonces 
$$ \integ[][]{\exp(f(x))f'(x)}{dx}=\exp(f(x))+C $$

\textbf{Demostración:}

$$ {(\exp(f(x))+C)'}={(\exp(f(x)))'}=\exp(f(x))f'(x) $$

\begin{flushright}
$\blacksquare$
\end{flushright}

\item[\Col •] Se demostrará que si $P(x)$ es cualquier polinomio en $x$, entonces $\lim_{x\to\infty}\frac{P(x)}{\exp(x)}=0$. Enseguida se mostrarán las ideas que ayudarán a llegar a esta conclusión.

\item[\Col •] Se probó que $\lim_{x\to\infty}\frac{x}{\exp(x)}=0$. Veamos que pasa con $\lim_{x\to\infty}\frac{x^2}{\exp(x)}$.

$$ \lim_{x\to\infty}\frac{x^2}{\exp(x)}=\lim_{x\to\infty}\frac{x^2}{\exp(\frac{x}{2}+\frac{x}{2})}=\lim_{x\to\infty}\frac{x^2}{\exp(\frac{x}{2})\exp(\frac{x}{2})}=\lim_{x\to\infty}\frac{x}{\exp(\frac{x}{2})}\frac{x}{\exp(\frac{x}{2})} $$
$$ =\lim_{x\to\infty}4\frac{\frac{x}{2}}{\exp(\frac{x}{2})}\frac{\frac{x}{2}}{\exp(\frac{x}{2})} $$

Sea $t=\frac{x}{2}$. Si $x\to \infty$, entonces $t\to \infty$, además, si $t\to \infty$, entonces $x\to \infty$, por lo tanto, $x\to \infty$, si y sólo si $t\to \infty$. De manera que

$$ \lim_{x\to\infty}4\frac{\frac{x}{2}}{\exp(\frac{x}{2})}\frac{\frac{x}{2}}{\exp(\frac{x}{2})}=\lim_{t\to\infty}4\frac{t}{\exp(t)}\frac{t}{\exp(t)}=2^2\lim_{x\to\infty}\frac{t}{\exp(t)}\lim_{x\to\infty}\frac{t}{\exp(t)}=2^2\cdot 0\cdot 0=0 $$

\item[\Col •] Ahora veamos que ocurre con $\lim_{x\to\infty}\frac{x^3}{\exp(x)}$.

$$ \lim_{x\to\infty}\frac{x^3}{\exp(x)}=\lim_{x\to\infty}\frac{x^3}{\exp(\frac{x}{3}+\frac{x}{3})+\frac{x}{3}}=\lim_{x\to\infty}\frac{x^3}{\exp(\frac{x}{2})\exp(\frac{x}{2})\exp(\frac{x}{3})}$$
$$=\lim_{x\to\infty}\frac{x}{\exp(\frac{x}{3})}\frac{x}{\exp(\frac{x}{3})}\frac{x}{\exp(\frac{x}{3})}=\lim_{x\to\infty}27\frac{\frac{x}{3}}{\exp(\frac{x}{3})}\frac{\frac{x}{3}}{\exp(\frac{x}{3})}\frac{\frac{x}{3}}{\exp(\frac{x}{3})} $$

Sea $t=\frac{x}{3}$. Si $x\to \infty$, entonces $t\to \infty$, además, si $t\to \infty$, entonces $x\to \infty$, por lo tanto, $x\to \infty$, si y sólo si $t\to \infty$. De manera que

$$ \lim_{x\to\infty}27\frac{\frac{x}{3}}{\exp(\frac{x}{3})}\frac{\frac{x}{3}}{\exp(\frac{x}{3})}\frac{\frac{x}{3}}{\exp(\frac{x}{3})}=\lim_{t\to\infty}27\frac{t}{\exp(t)}\frac{t}{\exp(t)}\frac{t}{\exp(t)}$$
$$=3^3\lim_{x\to\infty}\frac{t}{\exp(t)}\lim_{x\to\infty}\frac{t}{\exp(t)}\lim_{x\to\infty}\frac{t}{\exp(t)}=3^3\cdot 0\cdot 0=0 $$

\item[\Col •] En general, se cumple que si $k=\frac{x}{t}$, entonces $\lim_{x\to\infty}\frac{x^k}{\exp(x)}=\lim_{t\to\infty}k^k\left(\frac{t}{\exp(t)}\right)^k$. Con esto podemos probar el teorema pendiente.

\item[\Col •] \textbf{Teorema:} Si $P(x)$ es cualquier polinomio en $x$, entonces 
$$ \lim_{x\to\infty}\frac{P(x)}{\exp(x)}=0 $$

\textbf{Demostración:} 

$$ \limf{x}\frac{P(x)}{\exp(x)}=\limf{x}\frac{a_0+a_1x+a_2x^2+\cdots a_{n-1}x^{n-1}+a_nx^n}{\exp(x)} $$
$$ =\limf{x}\left( a_0\frac{1}{\exp(x)}+a_1\frac{x}{\exp(x)} a_2\frac{x^2}{\exp(x)}+\cdots+a_n\frac{x^n}{\exp(x)} \right)$$
$$=\limf{x}\left( a_0\frac{1}{\exp(x)}+\suma{i=1}{n}{a_i(i)^{i}\left(\frac{\frac{x}{i}}{\exp(\frac{x}{i})}\right)^{i}} \right)=a_0(0)+a_1(0)+a_2(0)+\cdots+a_{n-1}(0)+a_n(0)=0 $$

\begin{flushright}
$\blacksquare$
\end{flushright}

\end{itemize}

\subsection{\Col Nueva notación para la exponencial}

\begin{itemize}
\item[\Col •] Se probará que $\exp(x)=e^x$. Esto se cumple en todos los casos, por ejemplo:

\begin{enumerate}
\item[a)] $\exp(1)=e$.
\item[b)] $\exp(2)=\exp(1+1)=\exp(1)\exp(1)=e\cdot e=e^2$.
\item[c)] $\exp(n)=\exp(\underbrace{1+1+\cdots+1}_{\mbox{n veces}})=\underbrace{\exp(1)\exp(1)\cdots\exp(1)}_{\mbox{n veces}}=\exp(1)^n=e^n$.
\item[d] $\exp(-1)=\frac{1}{\exp(1)}=\exp(1)^{-1}=e^{-1}$.
\item[e)] $\exp(-n)=\frac{1}{\exp(n)}=\frac{1}{e^n}=e^{-n}$.
\end{enumerate}

\item[\Col •] \textbf{Teorema:} $\exp(x)=e^{x}\quad\paratodoxen{x}{\mathbb{R}}$.

\textbf{Demostración:} 

$$ \exp(x)=\exp(\underbrace{1+1+\cdots+1}_{\mbox{$x$ veces}})=\underbrace{\exp(1)\exp(1)\cdots\exp(1)}_{\mbox{$x$ veces}}=\underbrace{e\cdot e\cdots e}_{\mbox{$x$ veces}}=e^x.$$

\begin{flushright}
$\blacksquare$
\end{flushright}

\item[\Col •] \textbf{Teorema:} $\exp(-x)=e^{-x} \quad \forall x\in\mathbb{R}$

\textbf{Demostración:}

$$ \exp(-x)=\frac{1}{\exp(x)}=\frac{1}{e^{x}}=(e^x)^{-1}=e^{-x} $$

\begin{flushright}
$\blacksquare$
\end{flushright}

\item[\Col •] \textbf{Teorema:} $\left(\exp\left(\frac{1}{n}\right)\right)^m \paratodoxen{n,m}{R}$

\textbf{Demostración:} 

$$ \left(\exp\left(\frac{1}{n}\right)\right)^m=\underbrace{\exp\left(\frac{1}{n}\right)\cdot\exp\left(\frac{1}{n}\right)\cdots\exp\left(\frac{1}{n}\right)}_{\mbox{$m$ veces}} $$
$$ =\exp\left( \underbrace{\frac{1}{n}+\frac{1}{n}+\cdots+\frac{1}{n}}_{\mbox{$m$ veces}} \right)=\exp\left( m\left(\frac{1}{n}\right) \right)=\exp\left( \frac{m}{n} \right)=e^{\frac{m}{n}} $$

\begin{flushright}
$\blacksquare$
\end{flushright}

\item[\Col •] En conclusión, $\exp(x)=e^x \paratodoxen{x}{R}$.

\end{itemize}

\subsection{\Col Leyes de los exponentes}

\begin{itemize}

\item[\Col •] \textbf{Proposición:} \textbf{(Leyes de los exponentes)} $\forall a,b\in\mathbb{R}$ y $\forall r\in\mathbb{Q}$ ocurre que:
\begin{enumerate}
\item[a)] $e^a\cdot e^b=e^{a+b}$.
\item[b)] $\frac{e^a}{e^b}=e^{a-b}$.
\item[c)] $({e^a})^{r}=e^{ar}$.
\end{enumerate}

\textbf{Demostración:}

\begin{enumerate}
\item[a)] Por demostrar que $e^a\cdot e^b=e^{a+b}$.
$$e^a\cdot e^b=\exp(a)\exp(b)=\exp(a+b)=e^{a+b}.$$
\item[b)] Por demostrar que $\frac{e^a}{e^b}=e^{a-b}$.
$$\frac{e^a}{e^b}=\frac{\exp(a)}{\exp(b)}=\exp(a-b)=e^{a-b}.$$
\item[b)] Por demostrar que $({e^a})^{r}=e^{ar}$.
$$({e^a})^{r}=e^{ar}=\exp(a)^{r}=\exp(ra)=\exp(ar)=e^{ar}.$$

\begin{flushright}
$\blacksquare$
\end{flushright}

\end{enumerate}

\end{itemize}

\subsection{\Col Cambio de base}

\begin{itemize}
\item[\Col •] Se observa que $x\ln(a)=\ln(a^x)$, entonces $\exp(x\ln(a))=\exp(\ln(a^x))=a^x$, pues la función exponencial es la inversa de la función logaritmo. Esta última igualdad da lugar a la siguiente definición:

\item[\Col •] \textbf{Definición:} Sea $a>0$ una constante. Definimos 
$$a^x=\exp(x\ln(a)).$$

\begin{flushright}
$\square$
\end{flushright}

\item[\Col •] \textbf{Observación:} $1^x=\exp(x\ln(1))=\exp(x\cdot 0)=\exp(0)=1 \quad \paratodoxen{x}{R}$.
 
\item[\Col •] Sea $a>0$, entonces la función $f(x)=a^x$ es diferenciable y $\derivate{x}{f(x)}=\derivate{x}{a^x}=\derivate{x}{\exp(x\ln(a))}=\exp(x\ln(a))\derivate{x}{(x\ln(a))}=\exp(x\ln(a))(\ln(a))=a^x\ln(a)$. De lo que podemos concluir que, si $f(x)=a^x$, entonces $\derivate{x}{f(x)}=a^x\ln(a)$.

\item[\Col •] \textbf{Teorema:} Sea $a>0$ y $a\neq 1$, entonces $\integ[][]{a^x}{dx}=\frac{a^x}{\ln(a)}+C$.

\textbf{Demostración:}

Se observó que si $f(x)=a^x$, entonces $\derivate{x}{f(x)}=a^x\ln(a)$, por lo tanto $\integ[][]{a^x\ln(a)}{dx}=a^x+C$, de manera que $\ln(a)\integ[][]{a^x}{dx}=a^x+C$, se sigue que 
$$\integ[][]{a^x}{dx}=\frac{a^x}{\ln(a)}+\frac{C}{\ln(a)}=\frac{a^x}{\ln(a)}+C'$$  
 
\begin{flushright}
$\blacksquare$
\end{flushright} 
 
\end{itemize}

\subsection{\Col La gráfica de la función $a^x$ con $a$ fijo}

\begin{itemize}
\item[\Col •] A continuación se trazará con todo detalle la gráfica de la función $f(x)=a^x$ enunciando todas sus propiedades: 0) Dominio, imagen y puntos por los que pasa la función; 1) Primera derivada, intervalos de crecimiento y decrecimiento, y puntos críticos estacionarios o puntos singulares; 2) Segunda derivada, intervalos de concavidad positiva y negativa, y puntos de inflexión; 3) Límites relevantes y comportamiento asintótico. 

\begin{enumerate}
\item[0)] $Dom(a^x)=(-\infty,\infty)$, $Im(a^x)=(0,\infty)$ y Pasa por los puntos $(0,1)$ y $(1,a)$.
\item[1)] $$ \derivate{x}{a^x}=a^x\ln(a)
\left\{
 \begin{array}{lll}
  <0  & \mbox{si } 0<a<1 \\
  \\ =0 & \mbox{si } a=1 \\
  \\ >0 & \mbox{si } 1<a \\
 \end{array}
\right.$$

Por lo tanto:

\begin{enumerate}

\item[\textbf{Caso 1.}]  $0<a<1$
\begin{enumerate}
\item Intervalo de crecimiento$=\varnothing$.
\item Intervalo de decrecimiento$=(-\infty,\infty)$.
\item No existen puntos críticos estacionarios.
\end{enumerate} 

\item[\textbf{Caso 2.}]  $a=1$
\begin{enumerate}
\item Intervalo de crecimiento$=\varnothing$.
\item Intervalo de decrecimiento$=\varnothing$.
\item No existen puntos críticos estacionarios.
\end{enumerate} 

\item[\textbf{Caso 3.}]  $1<a$
\begin{enumerate}
\item Intervalo de crecimiento$=(-\infty,\infty)$.
\item Intervalo de decrecimiento$=\varnothing$.
\item No existen puntos críticos estacionarios.
\end{enumerate} 

\end{enumerate}

\item[2)] $$ \frac{\mathrm{d^2}}{\mathrm{d}x} \left(  {a^x}  \right)  
\left\{
 \begin{array}{ll}
  >0  & \mbox{si } a\neq 1 \\
  \\ =0 & \mbox{si } a=1 \\
 \end{array}
\right.$$

Por lo tanto:

\begin{enumerate}

\item[\textbf{Caso 1.}]  $0<a<1$
\begin{enumerate}
\item Intervalo de concavidad positiva$=(-\infty,\infty)$.
\item Intervalo de concavidad negativa$=\varnothing$.
\item No existen puntos de inflexión.
\end{enumerate} 

\item[\textbf{Caso 2.}]  $a=1$
\begin{enumerate}
\item Intervalo de concavidad positiva$=\varnothing$.
\item Intervalo de concavidad negativa$=\varnothing$.
\item No existen puntos de inflexión.
\end{enumerate} 

\item[\textbf{Caso 3.}]  $1<a$
\begin{enumerate}
\item Intervalo de concavidad positiva$=(-\infty,\infty)$.
\item Intervalo de concavidad negativa$=\varnothing$.
\item No existen puntos de inflexión.
\end{enumerate} 

\end{enumerate}

\item[3)] Límites relevantes 

\begin{enumerate}

\item[\textbf{Caso 1.}]  $0<a<1$
\begin{enumerate}
\item $\limf{x}a^x=\limf{x}\exp(x\ln(a))=0$.
\item $\limit{x}{-\infty}a^x=\limit{x}{-\infty}\exp(x\ln(a))=\infty$.
\end{enumerate} 

\item[\textbf{Caso 2.}]  $a=1$
\begin{enumerate}
\item $\limf{x}a^x=\limf{x}\exp(x\ln(a))=\limf{x}\exp(x\cdot 0)=1$.
\item $\limit{x}{-\infty}a^x=\limit{x}{-\infty}\exp(x\ln(a))=\limit{x}{-\infty}\exp(x\cdot 0)=1$.
\end{enumerate} 

\item[\textbf{Caso 3.}]  $1<a$
\begin{enumerate}
\item $\limf{x}a^x=\limf{x}\exp(x\ln(a))=\infty$.
\item $\limit{x}{-\infty}a^x=\limit{x}{-\infty}\exp(x\ln(a))=0$.
\end{enumerate} 

\end{enumerate}

\end{enumerate}

\item[\Col •] Dado lo anterior, la gráfica de la función $f(x)=a^x$ tiene el siguiente aspecto:

$$ incluir grafico $$

\item[\Col •] \textbf{Teorema:} Si $1<a<b$, entonces 

\begin{enumerate}
\item[a)] $a^x<b^x$ si $x>0$.
\item[b)] $a^x=b^x$ si $x=0$.
\item[c)] $a^x>b^x$ si $x<0$.
\end{enumerate}

\textbf{Demostración:} 

Sea $\frac{a^x}{b^x}=\frac{\exp(x\ln(a))}{\exp(x\ln(b))}=\exp(x\ln(a))\cdot\exp(x\ln(b))^{-1}=\exp(x\ln(a))\cdot\exp(-x\ln(b))=\exp(x\ln(a)-x\ln(b))=\exp(x(\ln(a)-\ln(b)))$, entonces $\frac{a^x}{b^x}=\exp(x[\ln(a)-\ln(b)])$. 

Si $r=\ln(a)-\ln(b)$, entonces $r<0$ y 

$$\frac{a^x}{b^x}=\exp(x\cdot r)
\left\{
 \begin{array}{lll}
  <1  & \mbox{si } x>0 & \mbox{, entonces} \quad a^x<b^x \\
  \\ =1 & \mbox{si } x=0 & \mbox{, entonces} \quad a^x=b^x \\
  \\ >1 & \mbox{si } x<0 & \mbox{, entonces} \quad a^x>b^x \\
 \end{array}
\right.$$

\begin{flushright}
$\blacksquare$
\end{flushright} 

\end{itemize}

$\linebreak$ 

\textbf{Teorema:} Si $0<a<b<1$, entonces
\begin{enumerate}
\item[i)] $a^x<b^x$ si $x>0$.
\item[ii)] $a^x=b^x$ si $x=0$. 
\item[iii)] $a^x>b^x$ si $x<0$.
\end{enumerate}

$\linebreak$

\textit{Demostración:} Sea $\frac{a^x}{b^x}=\exp(x[\ln(a)-\ln(b)])$. Si $r=\ln(a)-\ln(b)$, entonces $r<0$ y

$$ \frac{a^x}{b^x}=\exp(r\cdot x)\left\{
 \begin{array}{lll}
  <1  & \mbox{si } x>0 & \mbox{, entonces} \quad a^x<b^x \\
  \\ =1 & \mbox{si } x=0 & \mbox{, entonces} \quad a^x=b^x \\
  \\ >1 & \mbox{si } x<0 & \mbox{, entonces} \quad a^x>b^x \\
 \end{array}
\right..$$

\begin{flushright}
$\blacksquare$
\end{flushright}

En resumen, se definió la función $f(x)=a^x=\exp(x\ln(a))$ con $a>0$ constante, se estudio el caso especial $a=1$ y se demostró el aspecto del siguiente gráfico:

$$ incluir grafico $$

\subsection{\Col Algunas propiedades algebraicas}

$\linebreak$

\textbf{Teorema:} Sea $a$ constante, entonces
\begin{enumerate}
\item[i)] $a^xa^y=a^{x+y}$.
\item[ii)] $a^{-x}=\frac{1}{a^{-x}}$.
\item[iii)] $\frac{a^x}{a^y}=a^{x-y}$.
\item[iv)] $\left(a^x\right)^y=a^{xy}$.
\end{enumerate}

$\linebreak$

\textit{Demostración:} 

\begin{enumerate}
\item[i)] Por demostrar $a^xa^y=a^{x+y}$.

\begin{equation*}
\begin{split}
 a^xa^y & =\exp(x\ln(a))\cdot\exp(y\ln(a))=\exp(x\ln(a)+y\ln(a))=\exp(\ln(a)(x+y)) =\exp(\ln(a^{x+y}))\\
	& =a^{x+y}.
\end{split}
\end{equation*}
 
\item[ii)] Por demostrar  $a^{-x}=\frac{1}{a^{-x}}$.

\begin{equation*}
\begin{split}
 a^{-x} & =\exp(-x\ln(a))=\exp(x\ln(a))^{-1}=\frac{1}{\exp(\ln(a^x))}=\frac{1}{a^x}.
\end{split}
\end{equation*}

\item[iii)] Por demostrar $\frac{a^x}{a^y}=a^{x-y}$.

\begin{equation*}
\begin{split}
 \frac{a^x}{a^y} & =\frac{\exp(x\ln(a))}{\exp(y\ln(a)}=\exp(x\ln(a))\cdot\exp(y\ln(a))^{-1}=\exp(x\ln(a))\cdot\exp(-y\ln(a)) \\
 		& =\exp(x\ln(a)-y\ln(a))=\exp(\ln(a)(x-y))=\exp(\ln(a^{x-y})) \\
 		& =a^{x-y}.
\end{split}
\end{equation*}

\item[iv)] Por demostrar $\left(a^x\right)^y=a^{xy}$.

\begin{equation*}
\begin{split}
 \left( a^x \right)^y &=\exp(x\ln(a))^y=\exp(xy\ln(a))=\exp(\ln(a^{xy}))=a^{xy}.  \\
\end{split}
\end{equation*}

\begin{flushright}
$\blacksquare$
\end{flushright}

\end{enumerate}

$\linebreak$

\textbf{Teorema:} $ \limit{n}{\infty} n\left( \sqrt[n]{x}-1 \right)=\ln(x) $.

$\linebreak$

\textit{Demostración:} 

$$\limit{n}{\infty} n\left( \sqrt[n]{x}-1  \right) =\limit{n}{\infty} \frac{x^{\frac{1}{n}}-1}{\frac{1}{n}}. $$

Sea $b=\frac{1}{n}$, entonces $\limit{b}{0}\frac{x^b-1}{b}=\limit{b}{0}\frac{x^b-x^0}{b}=\limit{b}{0}\frac{\exp(b\ln(x)) -\exp(0\ln(x))}{b}$. Además, $\limit{b}{0}\frac{\exp(b\ln(x)) -\exp(0\ln(x))}{b}=\limit{b}{0}\frac{f(b)-f(0)}{b}=f'(0)$ donde $f(b)=\exp(b\ln(x))$. Entonces $\limit{n}{\infty} n\left( \sqrt[n]{x}-1  \right)=f'(0)=\ln(x)$.

\begin{flushright}
$\blacksquare$
\end{flushright}

\subsection{\Col Logaritmo base $a$ de $x$}

Se vio que $e^x=\exp(x)$ y que $a^x=\exp(x\ln(a))$, ahora se hablará de $\log_a(x)$. Pues el contenido anterior trató de $\ln(x)=\log_e(x)$. Se sabe que la función $f(x)=a^x$ tiene el siguiente aspecto para diferentes valores de $a$ tal que $a\in(0,1)\cup(1,\infty)$.

$$ incluir grafico $$

Para encontrar $\log_a(x)$ habrá que invertir $a^x \quad \forall x\in (0,1)\cup (1,\infty)$. Sea $a>0$ ($a\neq 1$). Se probó que $a^x:(-\infty,\infty)\to (0,\infty)$ es una función biyectiva por lo tanto es invertible.

$\linebreak$

\textbf{Definición:} \textbf{(Logaritmo base $a$ de $x$)} $\log_a(x):\left(0,\infty\right)\to\left(-\infty,\infty\right)$ se define como la función inversa de $a^x$, tal que $x=\log_a(y)$ si y sólo si $y=a^x$, de manera que $x=\log_a(a^x)$.

\begin{flushright}
$\square$
\end{flushright}

En particular, sucede que $a^{\log_a(x)}=x$, por lo que podemos decir que "el logaritmo base $a$ de $x$ es el exponente al cual hay que elevar $a$ para obtener $x$". Como $\log_{a}(x)$ es la función inversa de $a^x$, entonces se tiene que

$$ \log_a(a^x)=x=a^{\log_a(x)}. $$

Por definición, $a^{\log_a(x)}=x$, entonces, aplicando logaritmo natural en ambos lados de la desigualdad se obtiene que $\ln(a^{\log_a(x)})=\ln(x)$, por lo que $\log_a(x)\ln(a)=\ln(x)$, de modo que, despejando, se obtiene un manera sencilla de calcular $\log_a(x)$: 
$$\log_a(x)=\frac{1}{\ln(a)}\ln(x).$$ 

\textbf{Observación:} Como $\ln(e)=1$, entonces se tiene que
$$ \log_e(x)=\frac{1}{\ln(e)}\ln(x)=\ln(x) $$

\begin{flushright}
$\square$
\end{flushright}

\subsection{\Col Propiedades del logaritmo base $a$}

\textbf{Teorema: (Propiedades del logaritmo base $a$)} Sean $x,y>0$ y $r\in\mathbb{R}$, entonces 

\setlength{\columnsep}{-1.1in}
\begin{multicols}{2}
    \begin{itemize}
        \item[i)] $\log_a(xy)=\log_a(x)+\log_a(y)$,
        \item[ii)] $\log_a(x^{-1})=-\log_a(x)$,
        \item[iii)] $\log_a(\frac{x}{y})=\log_a(x)-\log_a(y)$,
        \item[iv)] $\log_a(x^r)=r\log_a(x)$.
    \end{itemize}
\end{multicols}

\textit{Demostración:} La demostración del teorema se realizará utilizando la definición de logaritmo natural base $a$ y usando las propiedades ya conocidas del logaritmo natural. 

\begin{enumerate}

\item[i)]
$$\begin{aligned}
 \log_a(xy) & = \frac{1}{\ln(a)}\ln(xy)=\frac{1}{\ln(a)}\left[ \ln(x)+\ln(y) \right]=\frac{1}{\ln(a)}\ln(x)+\frac{1}{\ln(a)}\ln(y) \\
 &  =\log_a(x)+\log_a(y). \\
\end{aligned}$$

\item[ii)]
$$ \log_a(x^{-1})=\frac{1}{\ln(a)}\ln(x^{-1})=-\frac{1}{\ln(a)}\ln(x)=-\log_a(x). $$

\item[iii)]
\begin{equation*}
\begin{split}
\log_a\left(\frac{x}{y}\right) & =\frac{1}{\ln(a)}\log\left(\frac{x}{y}\right)=\frac{1}{\ln(a)}\left[ \ln(x)-\ln(y) \right] \\ 
& =\frac{1}{\ln(a)}\ln(x)-\frac{1}{\ln(a)}\ln(y)=\log_a(x)-\log_a(y).
\end{split}
\end{equation*}

\item[iv)]
$$ \log_a\left(x^r\right)=\frac{r}{\ln(a)}\ln(x)=r\log_a(x). $$

\end{enumerate}

\begin{flushright}
$\blacksquare$
\end{flushright}

$\linebreak$

\subsection{\Col Límites relevantes para $\log_a(x)$}

\textbf{Teorema:} Sea $a\in\mathbb{R}$ tal que $a>0$, entonces 

\setlength{\columnsep}{-0.6in}
\begin{multicols}{2}
    \begin{itemize}
        \item[i)] $\limit{x}{\infty}\log_a(x)=\infty\quad$ si $\quad a>1$.
        \item[ii)] $\limit{x}{\infty}\log_a(x)=-\infty\quad$ si $\quad 0<a<1$.
        \item[iii)] $\limit{x}{0^+}\log_a(x)=-\infty\quad$ si $\quad a>1$.
        \item[iv)] $\limit{x}{0^+}\log_a(x)=\infty\quad$ si $\quad 0<a<1$.
    \end{itemize}
\end{multicols}

\textit{Demostración:}

\begin{enumerate}
\item[i)] Si $a>1$, entonces $\frac{1}{\ln(a)}>0$ de manera que 
$$ \limit{x}{\infty}\log_a(x)=\limit{x}{\infty}\frac{1}{\ln(a)}\ln(x)=\frac{1}{\ln(a)}\limit{x}{\infty}\ln(x)=\infty. $$
\item[ii)] Si $0<a<1$, entonces $\frac{1}{\ln(a)}<0$ de modo que 
$$ \limit{x}{\infty}\log_a(x)=\limit{x}{\infty}\frac{1}{\ln(a)}\ln(x)=\frac{1}{\ln(a)}\limit{x}{\infty}\ln(x)=-\infty. $$
\item[iii)] Si $a>1$, entonces $\frac{1}{\ln(a)}>0$ de manera que 
$$ \limit{x}{0^+}\log_a(x)=\limit{x}{0^+}\frac{1}{\ln(a)}\ln(x)=\frac{1}{\ln(a)}\limit{x}{0^+}\ln(x)=-\infty .$$
\item[iv)] Si $0<a<1$, entonces $\frac{1}{\ln(a)}<0$ de modo que 
$$ \limit{x}{0^+}\log_a(x)=\limit{x}{0^+}\frac{1}{\ln(a)}\ln(x)=\frac{1}{\ln(a)}\limit{x}{0^+}\ln(x)=\infty .$$
\end{enumerate}

\begin{flushright}
$\blacksquare$
\end{flushright}

\subsection{\Col La gráfica de la función $\log_a(x)$ con $a$ fijo}

A continuación se examinaran las propiedades de la función $f(x)=\log_a(x)$ para hallar esbozar su gráfica.

% \begin{multicols}{2}
%     \begin{enumerate}
%         \setcounter{enumi}{0}
%         \item Dominio e imagen
%         \item Imagen de puntos por los que pasa la \\ función.
%         \item Primera derivada
%         \item Intervalos de crecimiento \\ y decredimiento
%         \item Puntos críticos estacionarios o \\ singulares
%         \item Segunda de derivada
%         \item Intervalos de concavidad positiva y negativa
%         \item Puntos de inflexión
%         \item Limites relevantes
%         \item Comportamiento asintótico 
%     \end{enumerate}
% \end{multicols}

\subsubsection{Dominio e imagen}
$$Dom\left( \log_a(x) \right)=(0,\infty) \quad\quad\quad Img\left(\log_a(x)\right)=(-\infty,\infty)$$

\subsubsection{Imagen de puntos por los que pasa la función}

Pasa por $(1,0)$.

\subsubsection{Primera derivada}

$$\der{x}f(x)=\frac{1}{\ln(a)x}
\left\{
 \begin{array}{ll}
  >0  & \mbox{si } 1<a \\
  <0 & \mbox{si } 0<a<1 \\
 \end{array}
\right.$$

Entonces la función es creciente en el intervalo $(0,\infty)$ cuando $1<a$ y decreciente en el intervalo $(0,\infty)$ cuando $0<a<1$.

\subsubsection{Segunda derivada}
$$\derivada{2}{x^2}f(x)=-\frac{1}{\ln(a)x^2}
\left\{
\begin{array}{ll}
    >0 & \mbox{si } 0<a<1 \\
    <0 & \mbox{si } 1<a
\end{array}
\right.$$

Entonces la función tiene concavidad positica en el inertvalo $(0,\infty)$ cuando $0<a<1$ y tiene concavidad negativa en el intervalo $(0,\infty)$. En ningún caso se observan puntos de inflexión.

\subsubsection{Comportamiento límite}

$$
\limf{x}\log_a(x)=\left\{
\begin{array}{ll}
    \infty & \mbox{si } 1<a \\
    -\infty & \mbox{si } 0<a<1
\end{array}    
\right. 
\quad\quad 
\limit{x}{o^+} \log_a(x)=\left\{
\begin{array}{ll}
    -\infty & \mbox{si } 1<a \\
    \infty & \mbox{si } 0<a<1
\end{array}    
\right. 
$$

\subsubsection{Gráfica de la función}

Entonces la función tiene el siguiente aspecto:

$$incuir grafico$$

Se obervan los siguientes 4 casos:
\begin{itemize}
\item Si $1<a<b$ y $x>1$, entonces $$\log_a(x)=\frac{1}{\ln(a)}\ln(x)\geq \frac{1}{\ln(b)}\ln(x)=log_b(x).$$
\item Si $1<a<b$ y $0<x<1$, entonces $$\log_a(x)=\frac{1}{\ln(a)}\ln(x)\leq \frac{1}{\ln(b)}\ln(x)=log_b(x).$$
\item Si $0<a<b<1$ y $x>1$, entonces $$\log_a(x)=\frac{1}{\ln(a)}\ln(x)> \frac{1}{\ln(b)}\ln(x)=log_b(x).$$
\item Si $0<a<b<1$ y $0<x<1$, entonces $$\log_a(x)=\frac{1}{\ln(a)}\ln(x)< \frac{1}{\ln(b)}\ln(x)=log_b(x).$$    
\end{itemize}

Entonces se puede demostrar el espacto del siguiente gráfico

$$incluir grafico$$

En resumen se tiene el siguiente grafico:

$$incluir grafico$$

\textbf{Teorema:} Para $x>0$ se tiene que
$$ \int\!\ln(x)\mbox{ d}x=x\ln(x)-x+C $$

\textit{Demostración:} Se tiene que 

$$\begin{aligned}
    \der{x}\left\{x\ln(x)-x+C\right\} & = \der{x}\left\{x\ln(x)\right\} -1  \\
    & = \ln(x)+x\frac{1}{x}-1 \\ 
    & = \ln(x) \\ 
\end{aligned}$$

Por lo tanto se tiene que

$$x\ln(x)-x+C=\int\!\der{x}\left\{x\ln(x)-x+C\right\}\mbox{ d}x=\int\!\ln(x)\mbox{ d}x.$$

\begin{flushright}
    $\blacksquare$
\end{flushright}

\subsection{\Col Cambio de base}

Se hallará la respuesta a preguntas similares a la siguiente: ¿Cuánto es $\log_a(x)$ si se conoce el valor de $\log_b(x)$ y $a\neq b$ con $a\neq 1$ y $b\neq 1$? Se sabe que 

$$\log_a(x)=\frac{\ln(x)}{\ln(a)} \quad \mbox{y} \quad \log_b(x)=\frac{\ln(x)}{\ln(b)},$$
$$\Leftrightarrow \quad \ln(x)=\log_a(x)\ln(a) \quad \mbox{y} \quad \ln(x)=\log_b(x)\ln(b), $$
$$\Leftrightarrow \quad \log_a(x)\ln(a)=\log_b(x)\ln(b).$$

Por lo tanto

$$\log_a(x)= \frac{\ln(b)}{\ln(a)}\log_b(x).$$

En este caso, el cociente $\frac{\ln(b)}{\ln(a)}$ es el factor de cambio de base.

[Poner ejercicios ]

\section{\Col Funciones hiperbólicas}

\textbf{Definición: (Seno y coseno hiperbólico)} El seno hiperbólico es una función $senh:\mathbb{R}\rightarrow\mathbb{R}$ tal que 
$$senh(x)=\frac{e^x-e^{-x}}{2}\quad\quad\forall x\in\mathbb{R},$$
y el coseno hiperbólico es una función $cosh:\mathbb{R}\rightarrow\mathbb{R}$ tal que 
$$cosh(x)=\frac{e^x+e^{-x}}{2}\quad\quad\forall x\in\mathbb{R}.$$

\begin{flushright}
    $\square$
\end{flushright}

\noindent A partir de las definiciones anteriores de pueden definir el resto de las funciones hiperbólicas.
$$\begin{array}{lll}
    \tanh:\mathbb{R}\rightarrow\mathbb{R}\quad\quad\quad & \tanh(x)=\frac{\senh(x)}{\cosh(x)}=\frac{e^x-e^{-x}}{e^x+e^{-x}}\quad\quad\quad & \forall x\in\mathbb{R} \\
    \\
    \coth:\mathbb{R}\rightarrow\mathbb{R}\quad\quad\quad & \coth(x)=\frac{\cosh(x)}{\senh(x)}=\frac{e^x+e^{-x}}{e^x-e^{-x}}\quad\quad\quad & \forall x\in\mathbb{R} \\
    \\
    \sech:\mathbb{R}\rightarrow\mathbb{R}\quad\quad\quad & \sech(x)=\frac{1}{\cosh(x)}=\frac{2}{e^x+e^{-x}}\quad\quad\quad & \forall x\in\mathbb{R} \\
    \\
    \csch:\mathbb{R}\rightarrow\mathbb{R}\quad\quad\quad & \csch(x)=\frac{1}{\senh(x)}=\frac{2}{e^x-e^{-x}}\quad\quad\quad & \forall x\in\mathbb{R} \\
\end{array}$$

% Incluir la motivación de estas funciones

Las analogias entre el mundo trigonométrico y el mundo hiperbólico son diversas. A continuación se muestran algunas propiedades.

\hspace{1cm}

\begin{center}
{\tabulinesep=1.2mm
\begin{tabu} {|c|c|}
    \hline
    \textbf{Mundo hiperbólico} &  \textbf{Mundo trigonométrico} \\ 
    \hline
    $\der{x}\senh(x)=\cosh(x)$ & $\der{x}\sen(x)=\cos(x)$ \\
    \hline
    $\der{x}\cosh(x)=\senh(x)$ & $\der{x}\cos(x)=\sen(x)$ \\ [0.5ex]
    \hline
    $\cosh(x)^2-\senh(x)^2=1$ & $\cosh(x)^2+\senh(x)^2=1$ \\
    \hline
$\senh(x\pm y)=\senh(x)\cosh(y)\pm \senh(y)\cosh(x)$ & $\begin{array}{c}
    \sen(x\pm y)=\cos(x)\sen(y)\pm \cos(y)\sen(x) \\ 
    \cos(x\pm y)=\cos(x)\cos(y)\mp \sen(y)\sen(x) \\
\end{array}$ \\
    \hline
    Ley de De Moivre & \\
    $\left(\cosh(x)+\senh(x)\right)^2=\cosh(nx)+\senh(nx)$ & $\left(\cos(x)+i\sen(x)\right)^2=\cos(nx)+i\sen(nx)$ \\
    \hline
\end{tabu}}
\end{center}

% TODO: incluir demostracion%

\hspace{1cm}

A continuacón de muestran algunas antderivadas de las funciones del mundo hiperbólico y del mundo trigonométrico:

\hspace{1cm}

% TODO: poner los resultados en stand by.
\begin{center}
{\tabulinesep=1.2mm
\begin{tabu} {|c|c|}
    \hline
    \textbf{Mundo hiperbólico} &  \textbf{Mundo trigonométrico} \\
    $\integ{\senh(x)}{dx}=\cosh(x)+c$ & $\integ{\sen(x)}{dx}=-\cos(x)+c$ \\
    \hline
    $\integ{\cosh(x)}{dx}=\senh(x)+c$ & $\integ{\cos(x)}{dx}=\sen(x)+c$ \\
    \hline
    $\integ{\tanh(x)}{dx}=\ln|\cosh(x)|+c$ & $\integ{\tan(x)}{dx}=-\ln|\cos(x)|+c$ \\
    \hline
    $\integ{\coth(x)}{dx}=\ln|\senh(x)|+c$ & $\integ{\cot(x)}{dx}=\ln|\sen(x)|+c$ \\
    \hline
    $\integ{\sech(x)}{dx}=standby$ & $\integ{\sec(x)}{dx}=\ln|\sec(x)+\tan(x)|+c$ \\
    \hline
    $\integ{\csch(x)}{dx}=standby$ & $\integ{\csc(x)}{dx}=\ln|\csc(x)-\cot(x)|+c$ \\
    \hline
\end{tabu}}
\end{center}

\noindent\textbf{Teorema:} Para toda $x\in\mathbb{R}$ se tiene lo siguiente:

\setlength{\columnsep}{-0.6in}
\begin{multicols}{2}
    \begin{enumerate}
        \item $\der{x}\tanh(x)=\sech(x)^2$
        \item $\der{x}\sech(x)=-\sech(x)\tanh(x)$
        \item $\der{x}\coth(x)=-\csch(x)^2$
        \item $\der{x}\csch(x)=-\csch(x)\coth(x)$
    \end{enumerate}
\end{multicols}

% TODO: Demostrar el teorema anterior. Algunas demostraciones se encuentran en la tarea.

\subsection{\Col Gráficas de las funciones hiperbólicas}

\subsubsection{$\senh(x)$}

\begin{enumerate}
\item $Dom(\senh)=(-\infty, \infty)$, $Img(\senh)=(-\infty, \infty)$ y pasa por el punto $(0,0)$. Es función impar, es decir, $\senh(-x)=-\senh(x)$.
\item $\der{x}\senh(x)=\cosh(x)=\frac{e^x+e^{-x}}{2}>0\quad\forall x\in\mathbb{R} \Rightarrow IC=(-\infty, \infty)$, $ID=\emptyset$. No existen puntos críticos estacionarios.
\item $$\derivada{2}{x^2}\senh(x)=\der{x}\cosh(x)=\senh(x)=\frac{e^x-e^{-x}}{2}=\left\{
    \begin{array}{ll}
        <0 & $si $ x<0 \Rightarrow IPC=(0,\infty)\\
        =0 & $si $ x=0 \Rightarrow ICN=(-\infty,0)\\
        >0 & $si $ x>0 \Rightarrow PI=(0,0)\\
    \end{array}    
    \right. $$
\item $$\limf{x}\senh(x)=\limf{x}\frac{e^x-e^{-x}}{2}=\infty \quad\quad \lim_{x\to-\infty}\senh(x)=\lim_{x\to-\infty}\frac{e^x-e^{-x}}{2}=-\infty$$
\end{enumerate}

Entonces la gráfica de la función $\senh(x)$ tiene el siguiente aspecto:

% TODO: Poner la grafica que se debe de poner
$$incluir grafico$$

\end{document}